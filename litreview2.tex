% DefCrypt main.tex
% Started 4/2/2017

\documentclass[11pt,twoside]{article}

% ==================================================================
% Subfile Package/Commands
% ==================================================================

\usepackage{subfiles,algorithm2e}

\newcommand{\onlyinsubfile}[1]{#1}
\newcommand{\notinsubfile}[1]{}

\makeatletter
\def\input@path{{./}{../../}}
\makeatother

% ==================================================================
% Crypto Packages
% ==================================================================

\usepackage{defcrypt}
\usepackage{cryptocode}


% ==================================================================
% Document
% ==================================================================

\begin{document}
    \renewcommand{\onlyinsubfile}[1]{}
    \renewcommand{\notinsubfile}[1]{#1}
  
\section{Syntax}

An \emph{All-Or-Nothing-Transform} $\schemefont{AONT}$ specifies two algorithms $(\schemefont{AONT.Transform}, \schemefont{AONT.Inverse})$, and a block length $\schemefont{AONT.bl}$. We have that $\schemefont{AONT.Transform}: \{ \bits^\schemefont{AONT.bl}\}^*\rightarrow \{ \bits^\schemefont{AONT.bl}\}^*$. We call the domain of this function ``message sequences'' and the range ``pseudo-message sequences''. Then $\schemefont{AONT.Inverse}$ is the inverse of this function, meaning that $\schemefont{AONT.Inverse}: \{ \bits^\schemefont{AONT.bl}\}^*\rightarrow \{ \bits^\schemefont{AONT.bl}\}^*$, a mapping that only needs to be defined on pseudo-message sequences that can be generated by $\schemefont{AONT.Transform}$. $\schemefont{AONT.Transform}$ can (and should) be randomized, while $\schemefont{AONT.Inverse}$ is not randomized.

The correctness condition for $\schemefont{AONT}$ is $$\Prob{\schemefont{AONT.Inverse}(\schemefont{AONT.Transform}((m_1,m_2\dots m_s))=(m_1,m_2,\dots m_s))} = 1$$ where the probability is taken over all possible message sequences $(m_1,m_2\dots m_s)$ and all possible randomness of the $\schemefont{AONT.Transform}$ function. 

\section{Rivest (1997)}

\section{Boyko (1999)/ Canetti et. al (2000)}

\section{To dos} 

\begin{itemize}
\item Prove that the package transform with OAEP/ OWFs work (explicitly) for the Rivest definition/ strong-Rivest definition
\item What is AONT used for and what kind of security do we need for that
\item What is the application I was thinking of and what kind of security do we need for that? 
\end{itemize}

\end{document}
